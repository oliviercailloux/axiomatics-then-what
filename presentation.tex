\RequirePackage[l2tabu, orthodox]{nag}
\RequirePackage{silence}
\documentclass[french,english]{beamer}
\input{preamble/packages}
\input{preamble/math_basics}
\input{preamble/math_mine}
\input{preamble/redac}
\input{preamble/draw}
\input{preamble/acronyms}

\setbeamertemplate{headline}[singleline]

\title{Axiomatics, then what?}
\subject{Social Choice}
\keywords{Axiomatization}
\author{Olivier Cailloux}
\institute[LAMSADE]{LAMSADE, Université Paris-Dauphine}
\date{\formatdate{30}{1}{2017}}

\begin{document}
\begin{frame}[plain]
	\tikz[remember picture,overlay]{
		\path (current page.south west) node[anchor=south west, inner sep=0] {
			\includegraphics[height=1cm]{LAMSADE95.jpg}
		};
		\path (current page.south) ++ (0, 1mm) node[anchor=south, inner sep=0] {
			\includegraphics[height=9mm]{Dauphine.jpg}
		};
		\path (current page.south east) node[anchor=south east, inner sep=0] {
			\includegraphics[height=1cm]{PSL.png}
		};
	}
	\titlepage
\end{frame}
\addtocounter{framenumber}{-1}

\section{Axiomatics}
\subsection{Axiomatics}
\begin{frame}
	\frametitle{\subsecname}
	\begin{quote}
		Rather than dream up a multitude of arbitration schemes and determine whether or not each withstands the best of plausibility in a host of special cases, let us invert the procedure. Let us examine our subjective intuition of fairness and formulate this as a set of precise desiderata that any acceptable arbitration scheme must fulfil. Once these desiderata are formalized as axioms, then the problem is reduced to a mathematical investigation of the existence of and characterization of arbitration schemes which satisfy the axioms.
	\end{quote}
	\citet[p. 121]{luce_games_1957}
\end{frame}

\subsection{Beyond axioms?}
\begin{frame}
	\frametitle{\subsecname}
	Possible view (that I want to counter):
	\begin{itemize}
		\item Mathematically capture the behavior of a rule (with axioms…)
		\item Left out for the user: confront their intuition about fairness with the axioms
		\item Scientific approach stops at the first step
		\item The rest is ultimately subjective
	\end{itemize}
	I propose a different view: 
	\begin{itemize}
		\item Confronting our subjective intuition of fairness with axioms is hard
		\item Nobody can do this
		\item Because we can’t compute implications
		\item We can help doing this scientifically
	\end{itemize}
\end{frame}

\section{The hard problem of reaching the equilibrium}
\subsection{Problem reduced?}
\begin{frame}
	\frametitle{\subsecname}
	\begin{itemize}
		\item Formulate our subjective intuition of fairness as axioms?
		\item \emph{Excluding} impossibilities?
		\item Making sure no hidden axiom is left out?
		\item Accept Arrow’s axioms?
		\item Who’s subjective intuition?
	\end{itemize}
\end{frame}

\subsection{Understand the rule?}
\begin{frame}
	\frametitle{\subsecname}
	\begin{itemize}
		\item When do you \emph{understand} a voting rule?
		\item Borda rule: I know how to count scores
		\item Is that all?
		\item Recall we want to capture our (?) idea of fairness
	\end{itemize}
\end{frame}

\subsection{What do axioms say?}
\begin{frame}
	\frametitle{\subsecname}
	\begin{itemize}
		\item Humans have limited deductive power
		\item Hence, knowing the definition of a rule does not determine whether we accept it
		\item Knowing the \emph{axiomatization} of a rule does not determine whether we accept it
		\item Wanted: equilibrium between principles and case-based, concrete intuitions \citep{goodman_fact_1983, rawls_theory_1999}
	\end{itemize}
\end{frame}

\subsection{What do axioms say? (Illustrations)}
\begin{frame}
	\frametitle{\subsecname}
	\begin{example}[Borda]
		\begin{itemize}
			\item We know Borda is the rule that satisfies neutrality, reinforcement, faithfulness, cancellation
			\item This does not obviously say that Borda fails on Condorcet
		\end{itemize}
	\end{example}
	\begin{example}[Dictatorship]
		\begin{itemize}
			\item One may want Arrow’s axioms
			\item But fail to see what it implies: dictatorship
			\item Once the implications are understood, one does not want all of Arrow’s axioms any more
		\end{itemize}
	\end{example}
\end{frame}

\subsection{Fishburn-against-Condorcet}
\begin{frame}[fragile]
	\frametitle{\subsecname}
	\begin{minipage}{5.5cm}
		\citet[p. 544]{fishburn_paradoxes_1974} argument against the Condorcet principle (see also \url{http://rangevoting.org/FishburnAntiC.html}).
		\begin{block}{Condorcet winner}
%		\setlength\abovedisplayskip{0 ex}
%		\begin{equation}
			$w \text{ VS } \mu, \mu \in \set{a, \ldots, h}\text{? }\uncover<2->{51/101}$
%		\end{equation}
		\end{block}
	\end{minipage}%
	\begin{minipage}{\columnwidth-5cm}
		\small
		\begin{equation}
			\begin{array}{lrrrrrr}
				&\multicolumn{6}{c}{\text{nb voters}}\\
			\cmidrule{2-7}
					&31	&19	&10	&10	&10	&21	\\
			\midrule
				1	&a	&a	&f	&g	&h	&h	\\
				2	&b	&b	&w	&w	&w	&g	\\
				3	&c	&c	&a	&a	&a	&f	\\
				4	&d	&d	&h	&h	&f	&w	\\
				5	&e	&e	&g	&f	&g	&a	\\
				6	&w	&f	&e	&e	&e	&e	\\
				7	&g	&g	&d	&d	&d	&d	\\
				8	&h	&h	&c	&c	&c	&c	\\
				9	&f	&w	&b	&b	&b	&b	\\
			\end{array}
		\end{equation}
	%	\begin{tikzpicture}
	%		\path node[profile matrix] (profile) {
	%				&31	&19	&10	&10	&10	&21	\\
	%			1	&a	&a	&f	&g	&h	&h	\\
	%			2	&b	&b	&w	&w	&w	&g	\\
	%			3	&c	&c	&a	&a	&a	&f	\\
	%			4	&d	&d	&h	&h	&f	&w	\\
	%			5	&e	&e	&g	&f	&g	&a	\\
	%			6	&w	&f	&e	&e	&e	&e	\\
	%			7	&g	&g	&d	&d	&d	&d	\\
	%			8	&h	&h	&c	&c	&c	&c	\\
	%			9	&f	&w	&b	&b	&b	&b	\\
	%		};
	%	\end{tikzpicture}
	%	BUT ranks:
	\end{minipage}
	\vspace{-1pt}
	\onslide<3>
	\begin{equation}
		\vspace*{-8.4pt}
		\begin{array}{lrrrrrrrrr}
			&\multicolumn{9}{c}{\text{ranks}}\\
		\cmidrule{2-10}
			&1	&≤2	&≤3	&≤4	&≤5	&≤6	&≤7	&≤8	&≤9	\\
		\midrule
		w	&0	& 30	& 30	& 51	& 51	& 82	& 82	& 82	&101	\\
		a	&50	& 50	& 80	& 80	& 101	& 101	& 101	&101	&101	\\
		\end{array}
	\end{equation}
\end{frame}

\section{Possible approach}
\subsection{Possible approach}
\begin{frame}
	\frametitle{\subsecname}
	\begin{itemize}
		\item To know whether one accepts a voting rule, we have to check whether one accepts the implications of the voting rule
		\item Can’t be done exhaustively
		\item Can be done using known possible problematic cases
		\item Axiomatics can help in providing reasonings
		\item Different axiomatics may have different convincing power
		\item Acceptance of people in concrete cases can in principle be studied empirically \citep{gaertner_empirical_2012}
	\end{itemize}
	\begin{block}{Proposition}
		\begin{itemize}
			\item Proposition: propose a research program aiming at such a study / or
			\item Propose first steps (a framework?)
		\end{itemize}
	\end{block}
\end{frame}

\subsection{Key element}
\begin{frame}
	\frametitle{\subsecname}
	\begin{itemize}
		\item Obviously, in many cases, no single decisive rule is the most appropriate
		\item We have to allow for incompleteness
		\item E.g. Impossible to completely rank all universities by “quality”
		\item Possible approach: search for non decisive rules (possible winners…)
	\end{itemize}
\end{frame}

\begin{frame}[plain]
	\addtocounter{framenumber}{-1}
	\begin{center}
		\huge
		\textit{Thank you for your attention!}
	\end{center}
\end{frame}

\appendix
\AtBeginSection{
}

\section{Bibliography}
\begin{frame}[allowframebreaks]
	\frametitle{\secname}
	\def\newblock{\hskip .11em plus .33em minus .07em}
 	\bibliography{zotero,philo-eco,beliefs}
\end{frame}

\section{Voting rules}
\subsection{Definition}
\begin{frame}
	\frametitle{\subsecname}
	\begin{itemize}
		\item Voting rule: a systematic way of aggregating different opinions and decide
		\item Multiple reasonable ways of doing this
		\item Different voting rules have different interesting properties
		\item None satisfy all desirable properties
	\end{itemize}
\end{frame}

\begin{frame}[fragile]
	\frametitle{Voting rule}
	
	\begin{description}[Alternatives]
		\item[Alternatives] $\allalts = \set{a, b, c, d, \ldots}$
		\item[Voters] $\allvoters = \set{1, 2, \ldots}$
		\item[Profile] function $\prof$ from $\allvoters$ to linear orders on $\allalts$.
		\item[Voting rule] function $f$ mapping each $\prof$ to winners $\emptyset \subset \alts \subseteq \allalts$.
	\end{description}
	\vfill
	\begin{center}
		\begin{tikzpicture}
			\path node[profile matrix] (profile) {
				\prof(1)&
				\prof(2)
				\\
				| (profile11) | a&
				b
				\\
				b&
				a
				\\
				c&
				| (profile32) | c
				\\
			};
			\path ($(profile.south west)!.5!(profile.south east)$) ++ (0, -5mm) node {$\prof$};
			
			\path node[draw, rectangle, fit=(profile11) (profile32), outer xsep=2mm, outer ysep=1mm] (justprofile) {};
			\path (justprofile.east) ++ (2.5cm, 0) node[inner sep=0] (winners) {\mbox{} $\alts = \Set{a, b}$};
			\path[draw, ->] (justprofile.east) to[bend left=35] node[anchor=south] {$f$} (winners.west);

			\path[draw, decorate, decoration={brace, mirror}] (justprofile.south west) -- (justprofile.south east);
		\end{tikzpicture}
	\end{center}
\end{frame}

\subsection{Example of a profile}
\begin{frame}[fragile]
	\frametitle{\subsecname}
	\begin{center}
		$\begin{array}{lrrrrrr}
			&\multicolumn{6}{c}{\text{nb voters}}\\
		\cmidrule{2-7}
				&33	&16	&3	&8	&18	&22	\\
		\midrule
			1	&a	&b	&c	&c	&d	&e	\\
			2	&b	&d	&d	&e	&e	&c	\\
			3	&c	&c	&b	&b	&c	&b	\\
			4	&d	&e	&a	&d	&b	&d	\\
			5	&e	&a	&e	&a	&a	&a	\\
		\end{array}$
	\end{center}
	Who wins?
	\tikz[remember picture,overlay]{
		\path (current page.south east) ++ (0, 1em) node[anchor=south east, inner sep=0] {
			\tiny Thanks to \href{http://www.lamsade.dauphine.fr/~lang/}{Jérôme Lang}
		};
	}
	\begin{itemize}
		\item Most top-1: $a$
		\item $c$ is in the top 3 for everybody
		\item delete worst first, lowest nb of pref: $c$, $b$, $e$, $a$ ⇒ $d$
		\item delete worst first, from bottom: $a$, $e$, $d$, $b$ ⇒ $c$
		\item Borda: $b$
		\item Condorcet: $c$
	\end{itemize}
\end{frame}

\subsection{Two voting rules}
\begin{frame}
	\frametitle{Borda}
	
Given a profile $\prof$:
	\begin{itemize}
		\item score of $a \in \allalts$: number of alternatives it beats
		\item the highest scores win
	\end{itemize}
	
	\begin{equation}
		\prof =
		\begin{array}{rrrrr}
			a	&	a	&	a	&	b	&	b\\
			b	&	b	&	b	&	c	&	c\\
			c	&	c	&	c	&	a	&	a
		\end{array}
	\end{equation}
	\begin{itemize}
		\item score $a$ is\dots? \pause $2 + 2 + 2 = 6$
		\item score $b$ is $1 + 1 + 1 + 2 + 2 = 7$
		\item score $c$ is $1 + 1 = 2$
	\end{itemize}
	Winner: $b$.
\end{frame}

\begin{frame}
	\frametitle{Condorcet’s principle}
	\begin{block}{Condorcet’s principle}
		We ought to take the Condorcet winner as sole winner if it exists.
		\begin{itemize}
			\item $a$ \emph{beats} $b$ iff more than half the voters prefer $a$ to $b$.
			\item $a$ is a \emph{Condorcet winner} iff $a$ beats every other alternatives.
		\end{itemize}
	\end{block}
	\vfill
	\begin{equation}
		\prof =
		\begin{array}{rrrrr}
			a	&	a	&	a	&	b	&	b\\
			b	&	b	&	b	&	c	&	c\\
			c	&	c	&	c	&	a	&	a
		\end{array}
	\end{equation}
	 Who wins? \pause $a$
\end{frame}

\begin{frame}
	\frametitle{Condorcet’s principle and a voting rule}
	\begin{itemize}
		\item Condorcet’s principle does not define a voting rule. Why? \pause
		\item No winner is defined when no Condorcet winner
	\end{itemize}
	\begin{equation}
		\prof =
		\begin{array}{rrr}
			b	&	c	&	d\\
			c	&	d	&	b\\
			a	&	b	&	a\\
			d	&	a	&	c
		\end{array}
	\end{equation}
	\pause
	$a$ loses against $b$; $b$ against $d$; $c$ against $b$; $d$ against $c$
	\begin{itemize}
		\item Dodgson’s method (1876): candidates “closest” to being Condorcet winners {\tiny (in nb of swaps)}
	\end{itemize}
\end{frame}

\end{document}

\section{Frame template}
\subsection{}
\begin{frame}
	\frametitle{\subsecname}
	\begin{itemize}
		\item 
	\end{itemize}
\end{frame}

